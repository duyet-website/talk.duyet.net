% Copyright 2004 by Till Tantau <tantau@users.sourceforge.net>.
%
% In principle, this file can be redistributed and/or modified under
% the terms of the GNU Public License, version 2.
%
% However, this file is supposed to be a template to be modified
% for your own needs. For this reason, if you use this file as a
% template and not specifically distribute it as part of a another
% package/program, I grant the extra permission to freely copy and
% modify this file as you see fit and even to delete this copyright
% notice. 

\documentclass{beamer}

% There are many different themes available for Beamer. A comprehensive
% list with examples is given here:
% http://deic.uab.es/~iblanes/beamer_gallery/index_by_theme.html
% You can uncomment the themes below if you would like to use a different
% one:
%\usetheme{AnnArbor}
%\usetheme{Antibes}
%\usetheme{Bergen}
%\usetheme{Berkeley}
%\usetheme{Berlin}
%\usetheme{Boadilla}
%\usetheme{boxes}
%\usetheme{CambridgeUS}
%\usetheme{Copenhagen}
%\usetheme{Darmstadt}
\usetheme{default}
%\usetheme{Frankfurt}
%\usetheme{Goettingen}
%\usetheme{Hannover}
%\usetheme{Ilmenau}
%\usetheme{JuanLesPins}
%\usetheme{Luebeck}
%\usetheme{Madrid}
%\usetheme{Malmoe}
%\usetheme{Marburg}
%\usetheme{Montpellier}
%\usetheme{PaloAlto}
%\usetheme{Pittsburgh}
%\usetheme{Rochester}
%\usetheme{Singapore}
%\usetheme{Szeged}
%\usetheme{Warsaw}

\addtobeamertemplate{navigation symbols}{}{%
    \usebeamerfont{footline}%
    \usebeamercolor[fg]{footline}%
    \hspace{1em}%
    \insertframenumber/\inserttotalframenumber
}

\title{Chatbot: Semantic Parsing and Logical Forms}

% A subtitle is optional and this may be deleted
% \subtitle{Optional Subtitle}

\author{Van-Duyet LE}
% - Give the names in the same order as the appear in the paper.
% - Use the \inst{?} command only if the authors have different
%   affiliation.

% \institute[Universities of Somewhere and Elsewhere] % (optional, but mostly needed)
% {
%   \inst{1}%
%   Department of Computer Science\\
%   University of Somewhere
%   \and
%   \inst{2}%
%   Department of Theoretical Philosophy\\
%   University of Elsewhere}
% - Use the \inst command only if there are several affiliations.
% - Keep it simple, no one is interested in your street address.

\date{Ho Chi Minh, 05/2018}
% - Either use conference name or its abbreviation.
% - Not really informative to the audience, more for people (including
%   yourself) who are reading the slides online

\subject{NLP, Chatbot}
% This is only inserted into the PDF information catalog. Can be left
% out. 

% If you have a file called "university-logo-filename.xxx", where xxx
% is a graphic format that can be processed by latex or pdflatex,
% resp., then you can add a logo as follows:

% \pgfdeclareimage[height=0.5cm]{university-logo}{university-logo-filename}
% \logo{\pgfuseimage{university-logo}}

% Delete this, if you do not want the table of contents to pop up at
% the beginning of each subsection:
\AtBeginSubsection[]
{
  \begin{frame}<beamer>{Outline}
    \tableofcontents[currentsection,currentsubsection]
  \end{frame}
}

% Let's get started
\begin{document}

\begin{frame}
  \titlepage
\end{frame}

\begin{frame}{Outline}
  \tableofcontents
  % You might wish to add the option [pausesections]
\end{frame}

% Section and subsections will appear in the presentation overview
% and table of contents.
\section{Logical Forms and Denotations}

\subsection{Chatbot}

\begin{frame}{An Example Chatbot}

    \begin{example}
    
        \begin{table}[]
        \centering
        \caption{Lunch Ordering}
        \label{lunch-ordering}
        \begin{tabular}{clr}
          & \textbf{Utterance}                   & \textbf{Intent} \\
        M & Hello, can I help you?               & greeting        \\
        H & Yes, I'd like to have some luch      & askMenu         \\
        M & Would you like a starter?            & askStarter      \\
        H & Yes, I'd like a chicken soup, please & chooseStarter   \\
        M & Would you like anything to drink?    & askDrink        \\
        H & No, thanks                           & confirmation   
        \end{tabular}
        \end{table}
        
    \pause
    \end{example}
    
    
    Every chatbot needs \textbf{intent detection} and \textbf{entity extraction}.

\end{frame}

\begin{frame}{Is Your Bot Intelligent?}
    \begin{itemize}
        \item Intent detection and entity extraction are not sufficient to make your chatbots intelligent
        \pause
        \item They cannot answer questions such as:
        \begin{itemize}
            \item \textit{What is the tallest mountain in Vietnam?}
            \item \textit{What is the capital in Vietnam?}
            \item \textit{What is three plus three plus one?}
            \item \textit{Who is Obama?}
        \end{itemize}
    \end{itemize}
\end{frame}

\subsection{Two Main Components of a Chatbot System}

\begin{frame}{Two Main Components of a Chatbot System}
    \begin{enumerate}
        \item \textbf{AI Engine}: language understanding \& intent detection
        \item \textbf{Dialog Engine}: state machine that executes context-driven workflows with scope variables
    \end{enumerate}
\end{frame}


\section{Parsing Utterances}

\begin{frame}{Semantic Parsing}
A semantic parsing maps natural language utterances into an intermediate logical form, which is executed to produce a denotation.

\begin{block}{A simple arithmetic task}
    \begin{itemize}
        \item Utterance: "What is three plus four?"
        \item Logical form: (+ 3 4)
        \item Denotation: 7
    \end{itemize}
\end{block}

\begin{block}{A question answering task}
    \begin{itemize}
        \item Utterance: "What is the capital of Vietnam?"
        \item Logical form: (capital "Vietnam")
        \item Denotation: "Hanoi"
    \end{itemize}
\end{block}

\end{frame}

\begin{frame}{Semantic Parsing}

\begin{block}{A travel agent bot}
    \begin{itemize}
        \item Utterance: \textit{Show me flights to Hanoi leaving tomorrow}
        \item Logical form: \textit{(and (type flight) (destination Hanoi) (departureDate 2018.05.10))}
        \item Denotation: \textit{(list ...)}
    \end{itemize}
\end{block}

\end{frame}

\begin{frame}{Semantic Representations}
    \begin{itemize}
        \item Semantic representations are generally logical forms, which are
expressions in a fully specified, unambiguous artificial language.
        \item There are a variety of different formalisms:
            \begin{itemize}
                \item lambda calculus
                \item natural logics
                \item diagrammatic languages
                \item programming languages
                \item robot controller languages
                \item Grammar formalism based schemes:
                \begin{itemize}
                    \item Dependency Formalism
                    \item Combinatory Categorial Grammar (CCG)
                    \item Head-Phrase Structure Grammar (HPSG)
                    \item Lexicalized Tree Adjoining Grammar (LTAG)
                \end{itemize}
                \item database query languages
                \item knowledge-based query languages (SPARQL, etc.)
                \item lambda dependency-based compositional semantics
            \end{itemize}
    \end{itemize}
\end{frame}

\section{Logical Forms}

\begin{frame}{Logical Forms}
    \begin{block}{Logical Forms}
        A \textbf{logical form} is a hierarchical expression. Primitive logical forms
represent concrete values:
        \begin{itemize}
            \item (boolean true)
            \item (number 23)
            \item (string "Chao buoi sang")
            \item (fb:en.obama)
        \end{itemize}
    \end{block}
\end{frame}


\begin{frame}{Logical Forms}
    \begin{itemize}
        \item Logical forms can be constructed recursively by a function name
followed by arguments, which are themselves logical forms
        \begin{itemize}
            \item \texttt{(call + (number 3) (number 4))}
            \item \texttt{(call java.lang.Math.cos (number 0))}
            \item \texttt{(call if (call < (number 3) (number 4)) (string yes) (string no))}
            \item \texttt{(call .indexOf (string "Duyet dep trai") (string "dep"))}
        \end{itemize}
        \item We can execute a logical form and get the denotation (answer): 
        \begin{center}
            \centering
            \fbox{
              \parbox{.6\textwidth}{
                \centering
                \texttt{(call + (number 3) (number 4))}
              }
            }
            \\
            \Downarrow \\
            \fbox{
              \parbox{.2\textwidth}{
                \centering
                \texttt{(number 7)}
              }
            }
        \end{center}
    \end{itemize}
\end{frame}

\begin{frame}{Parsing Utterances to Logical Forms}
    \begin{itemize}
        \item \textbf{How to map a natural language utterance into a logical form?}
        \item The key framework is compositionality:
        \begin{itemize}
            \item The meaning of a full sentence is created by combining the meanings of its parts
            \item Meanings are represented by logical forms.
        \end{itemize}
        \item Classical and powerful approach: use a \textbf{formal grammar}.
    \end{itemize}
\end{frame}

\begin{frame}{Formal grammar}
A grammar is a set of rules which specify how to combine logical
forms to build more complex ones in a manner that is guided by the
natural language

\begin{example}
    \begin{itemize}
        \item \texttt{(rule \$Expr (\$PHRASE) (NumberFn))}
        \item \texttt{(rule \$Operator (plus) (ConstantFn (lambda y (lambda x (call +(var x)(var y))))))}
        \item \texttt{(rule \$Operator (times) (ConstantFn (lambda y (lambda x (call * (var x)(var y))))))}
        \item \texttt{(rule \$Partial ($Operator $Expr) (JoinFn forward))}
    \end{itemize}
\end{example}
\end{frame}

\begin{frame}{Parsing }
    \begin{itemize}
        \item Now, the utterance "What is three plus four?" should give the output (number 7)
        \item A longer sentence such as "What is three plus four times
two? should give two derivations:
        \begin{itemize}
            \item \texttt{(number 14)}
            \item \texttt{(number 11)}
        \end{itemize}
        \item A parser is an actual algorithm that takes the grammar and generates those derivations.
        \begin{itemize}
            \item INP: \texttt{What is three plus four?}
            \item OUT: \texttt{(derivation (formula (((lambda y (lambda x (call +
(var x) (var y)))) (number 4)) (number 3))) (value
(number 7)))}
        \end{itemize}
    \end{itemize}
\end{frame}

\begin{frame}{Parsing Utterances to Logical Forms}
    \begin{itemize}
        \item A given utterance might be consistent with multiple logical forms, creating ambiguity
        \item \texttt{INP: What is three plus four times two?}
        \item OUT:
        \begin{enumerate}
            \item \texttt{(derivation (formula (((lambda y (lambda x (call + (var
x) (var y)))) (((lambda y (lambda x (call * (var x) (var
y)))) (number 2)) (number 4))) (number 3))) (value
(number 11)))}
            \item \texttt{(derivation (formula (((lambda y (lambda x (call * (var
x) (var y)))) (number 2)) (((lambda y (lambda x (call +
(var x) (var y)))) (number 4)) (number 3)))) (value
(number 14)))}
        \end{enumerate}
    \end{itemize}
\end{frame}

\begin{frame}{Parsing Utterances to Logical Forms}
    \begin{itemize}
        \item Computational challenge: the number of candidate logical forms is
in general exponential in the length of the sentence.
        \item In the question \textit{What kind of system of government does the United States have?} (Berant et al., 2013):
        \begin{itemize}
            \item the phrase "United States" maps to 231 entities in the lexicon,
            \item the verb "have" maps to 203 binaries,
            \item the phrases "kind", "system", and "government" all map to many
different unary and binary predicates.
        \end{itemize}
    \end{itemize}
\end{frame}

\section{Learning for Parsing Utterances to Logical Forms}

\begin{frame}{Learning}
    \begin{itemize}
        \item Machine learning concerns the ability to generalize from a set of past
observations or experiences in a way that leads to improved
performance on a future task (T. Mitchel, 1997).
        \item A ML system has 3 integral pieces:
        \begin{enumerate}
            \item a \textbf{feature representation} of the data
            \item an \textbf{objective function}
            \item an algorithm for \textbf{optimizing} the objective function
        \end{enumerate}
    \end{itemize}
\end{frame}

\begin{frame}{Using Machine Learning to get Logical Forms}
    Using Machine Learning for Parsing Utterances to Logical Forms?
    
\end{frame}


% Placing a * after \section means it will not show in the
% outline or table of contents.
\section*{Summary}

\begin{frame}{Summary}
  \begin{itemize}
  \item
    Many existing chatbot systems have mostly used a swallow semantic
representation of text, disregarding significant meaning encoded in
human language
  \item
    Recent logical and statistical approaches have identified methods for
mapping utterances to meanings efficiently.
  \end{itemize}
  
  \end{frame}



% All of the following is optional and typically not needed. 
\appendix
\section<presentation>*{\appendixname}
\subsection<presentation>*{For Further Reading}

\begin{frame}[allowframebreaks]
  \frametitle<presentation>{For Further Reading}
    
    \begin{itemize}
        \item Jonathan Herzig and Jonathan Berant, "Neural Semantic Parsing over
Multiple Knowledge-bases", Proceedings of ACL, 2017
        \item Jonathan Berant et al., "Semantic Parsing on Freebase from
Question-Answer Pairs", Proceedings of EMNLP, 2013.
        \item Phuong Le-Hong and Duc-Thien Bui, "A Factoid Question Answering
System for Vietnamese", Proceedings of WWW Companion, 2018.
    \end{itemize}

\end{frame}

\end{document}


